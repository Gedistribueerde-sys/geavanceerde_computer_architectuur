\section{Theory}

\subsection{Point Cloud}
A \textbf{point cloud} is a collection of data points defined in a three-dimensional coordinate system.  
Each point represents a position in space, typically described by $(x, y, z)$ coordinates.  
Point clouds are commonly acquired using 3D scanners, LiDAR sensors, or photogrammetry systems.  
They are used in applications such as 3D modeling, robotics, mapping, and computer vision.

\subsection{Voxel}
A \textbf{voxel} (volumetric pixel) is the smallest unit of a 3D grid, similar to how a pixel is the smallest unit of a 2D image.  
Voxels divide 3D space into uniform cubes, allowing volumetric representations of shapes or environments.  
They are used in 3D reconstruction, simulation, gaming, and medical imaging.

\subsection{Pointcloud to Voxel Grid Filtering}
\textbf{Pointcloud to voxel grid filtering} converts an unstructured point cloud into a regular 3D grid.  
Space is divided into equal-sized voxels, and each point is assigned to its corresponding voxel.  
This reduces data density, removes redundant points, and creates a structured representation that is easier and faster to process for tasks such as downsampling and spatial queries.

\subsection{Morton Codes}
\textbf{Morton codes} (also called Z-order curves) are a method of encoding multi-dimensional coordinates into a single integer value.  
They work by bit-interleaving the binary coordinates (e.g., $x$, $y$, $z$).  
This encoding preserves spatial locality, making it useful for data structures such as octrees and for accelerating spatial queries on GPUs.

\subsection{CUDA Thrust Library}
The \textbf{CUDA Thrust library} is a parallel algorithms library for NVIDIA GPUs.  
It provides high-level abstractions similar to the C++ Standard Template Library (STL), including parallel sorting, scanning, reduction, and vector operations.  
Thrust simplifies GPU programming by offering ready-to-use, highly optimized parallel primitives.

\subsection{LAS Files}
\textbf{LAS files} are a standardized file format used for storing LiDAR point cloud data.  
The format supports 3D coordinates, intensity values, classification labels, GPS time, color information, and other metadata.  
LAS is widely used in geospatial applications, surveying, and remote sensing due to its efficiency and interoperability.

